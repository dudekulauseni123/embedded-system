\subsection{Seven-Segment Display}
This subsection illustrates the use of SPI Communication between the Vaman-ESP32
and the ARM Cortex-M4 subsystem onboard the Vaman-Pygmy by controlling a
seven-segment display via the ARM interface, where the GPIO pins are exposed by
the FPGA platform.

\subsubsection{Components}
\begin{table}[!ht]
    \centering
    %%%%%%%%%%%%%%%%%%%%%%%%%%%%%%%%%%%%%%%%%%%%%%%%%%%%%%%%%%%%%%%%%%%%%%
%%                                                                  %%
%%  This is the header of a LaTeX2e file exported from Gnumeric.    %%
%%                                                                  %%
%%  This file can be compiled as it stands or included in another   %%
%%  LaTeX document. The table is based on the longtable package so  %%
%%  the longtable options (headers, footers...) can be set in the   %%
%%  preamble section below (see PRAMBLE).                           %%
%%                                                                  %%
%%  To include the file in another, the following two lines must be %%
%%  in the including file:                                          %%
%%        \def\inputGnumericTable{}                                 %%
%%  at the beginning of the file and:                               %%
%%        \input{name-of-this-file.tex}                             %%
%%  where the table is to be placed. Note also that the including   %%
%%  file must use the following packages for the table to be        %%
%%  rendered correctly:                                             %%
%%    \usepackage[latin1]{inputenc}                                 %%
%%    \usepackage{color}                                            %%
%%    \usepackage{array}                                            %%
%%    \usepackage{longtable}                                        %%
%%    \usepackage{calc}                                             %%
%%    \usepackage{multirow}                                         %%
%%    \usepackage{hhline}                                           %%
%%    \usepackage{ifthen}                                           %%
%%  optionally (for landscape tables embedded in another document): %%
%%    \usepackage{lscape}                                           %%
%%                                                                  %%
%%%%%%%%%%%%%%%%%%%%%%%%%%%%%%%%%%%%%%%%%%%%%%%%%%%%%%%%%%%%%%%%%%%%%%



%%  This section checks if we are begin input into another file or  %%
%%  the file will be compiled alone. First use a macro taken from   %%
%%  the TeXbook ex 7.7 (suggestion of Han-Wen Nienhuys).            %%
\def\ifundefined#1{\expandafter\ifx\csname#1\endcsname\relax}


%%  Check for the \def token for inputed files. If it is not        %%
%%  defined, the file will be processed as a standalone and the     %%
%%  preamble will be used.                                          %%
\ifundefined{inputGnumericTable}

%%  We must be able to close or not the document at the end.        %%
	\def\gnumericTableEnd{\end{document}}


%%%%%%%%%%%%%%%%%%%%%%%%%%%%%%%%%%%%%%%%%%%%%%%%%%%%%%%%%%%%%%%%%%%%%%
%%                                                                  %%
%%  This is the PREAMBLE. Change these values to get the right      %%
%%  paper size and other niceties.                                  %%
%%                                                                  %%
%%%%%%%%%%%%%%%%%%%%%%%%%%%%%%%%%%%%%%%%%%%%%%%%%%%%%%%%%%%%%%%%%%%%%%

	\documentclass[12pt%
			  %,landscape%
                    ]{report}
       \usepackage[latin1]{inputenc}
       \usepackage{fullpage}
       \usepackage{color}
       \usepackage{array}
       \usepackage{longtable}
       \usepackage{calc}
       \usepackage{multirow}
       \usepackage{hhline}
       \usepackage{ifthen}

	\begin{document}


%%  End of the preamble for the standalone. The next section is for %%
%%  documents which are included into other LaTeX2e files.          %%
\else

%%  We are not a stand alone document. For a regular table, we will %%
%%  have no preamble and only define the closing to mean nothing.   %%
    \def\gnumericTableEnd{}

%%  If we want landscape mode in an embedded document, comment out  %%
%%  the line above and uncomment the two below. The table will      %%
%%  begin on a new page and run in landscape mode.                  %%
%       \def\gnumericTableEnd{\end{landscape}}
%       \begin{landscape}


%%  End of the else clause for this file being \input.              %%
\fi

%%%%%%%%%%%%%%%%%%%%%%%%%%%%%%%%%%%%%%%%%%%%%%%%%%%%%%%%%%%%%%%%%%%%%%
%%                                                                  %%
%%  The rest is the gnumeric table, except for the closing          %%
%%  statement. Changes below will alter the table's appearance.     %%
%%                                                                  %%
%%%%%%%%%%%%%%%%%%%%%%%%%%%%%%%%%%%%%%%%%%%%%%%%%%%%%%%%%%%%%%%%%%%%%%

\providecommand{\gnumericmathit}[1]{#1} 
%%  Uncomment the next line if you would like your numbers to be in %%
%%  italics if they are italizised in the gnumeric table.           %%
%\renewcommand{\gnumericmathit}[1]{\mathit{#1}}
\providecommand{\gnumericPB}[1]%
{\let\gnumericTemp=\\#1\let\\=\gnumericTemp\hspace{0pt}}
 \ifundefined{gnumericTableWidthDefined}
        \newlength{\gnumericTableWidth}
        \newlength{\gnumericTableWidthComplete}
        \newlength{\gnumericMultiRowLength}
        \global\def\gnumericTableWidthDefined{}
 \fi
%% The following setting protects this code from babel shorthands.  %%
 \ifthenelse{\isundefined{\languageshorthands}}{}{\languageshorthands{english}}
%%  The default table format retains the relative column widths of  %%
%%  gnumeric. They can easily be changed to c, r or l. In that case %%
%%  you may want to comment out the next line and uncomment the one %%
%%  thereafter                                                      %%
\providecommand\gnumbox{\makebox[0pt]}
%%\providecommand\gnumbox[1][]{\makebox}

%% to adjust positions in multirow situations                       %%
\setlength{\bigstrutjot}{\jot}
\setlength{\extrarowheight}{\doublerulesep}

%%  The \setlongtables command keeps column widths the same across  %%
%%  pages. Simply comment out next line for varying column widths.  %%
\setlongtables

\setlength\gnumericTableWidth{%
	121pt+%
	96pt+%
	70pt+%
0pt}
\def\gumericNumCols{3}
\setlength\gnumericTableWidthComplete{\gnumericTableWidth+%
         \tabcolsep*\gumericNumCols*2+\arrayrulewidth*\gumericNumCols}
\ifthenelse{\lengthtest{\gnumericTableWidthComplete > \linewidth}}%
         {\def\gnumericScale{1*\ratio{\linewidth-%
                        \tabcolsep*\gumericNumCols*2-%
                        \arrayrulewidth*\gumericNumCols}%
{\gnumericTableWidth}}}%
{\def\gnumericScale{1}}

%%%%%%%%%%%%%%%%%%%%%%%%%%%%%%%%%%%%%%%%%%%%%%%%%%%%%%%%%%%%%%%%%%%%%%
%%                                                                  %%
%% The following are the widths of the various columns. We are      %%
%% defining them here because then they are easier to change.       %%
%% Depending on the cell formats we may use them more than once.    %%
%%                                                                  %%
%%%%%%%%%%%%%%%%%%%%%%%%%%%%%%%%%%%%%%%%%%%%%%%%%%%%%%%%%%%%%%%%%%%%%%

\ifthenelse{\isundefined{\gnumericColA}}{\newlength{\gnumericColA}}{}\settowidth{\gnumericColA}{\begin{tabular}{@{}p{121pt*\gnumericScale}@{}}x\end{tabular}}
\ifthenelse{\isundefined{\gnumericColB}}{\newlength{\gnumericColB}}{}\settowidth{\gnumericColB}{\begin{tabular}{@{}p{96pt*\gnumericScale}@{}}x\end{tabular}}
\ifthenelse{\isundefined{\gnumericColC}}{\newlength{\gnumericColC}}{}\settowidth{\gnumericColC}{\begin{tabular}{@{}p{70pt*\gnumericScale}@{}}x\end{tabular}}

\begin{tabular}[c]{%
	b{\gnumericColA}%
	b{\gnumericColB}%
	b{\gnumericColC}%
	}

%%%%%%%%%%%%%%%%%%%%%%%%%%%%%%%%%%%%%%%%%%%%%%%%%%%%%%%%%%%%%%%%%%%%%%
%%  The longtable options. (Caption, headers... see Goosens, p.124) %%
%	\caption{The Table Caption.}             \\	%
% \hline	% Across the top of the table.
%%  The rest of these options are table rows which are placed on    %%
%%  the first, last or every page. Use \multicolumn if you want.    %%

%%  Header for the first page.                                      %%
%	\multicolumn{3}{c}{The First Header} \\ \hline 
%	\multicolumn{1}{c}{colTag}	%Column 1
%	&\multicolumn{1}{c}{colTag}	%Column 2
%	&\multicolumn{1}{c}{colTag}	\\ \hline %Last column
%	\endfirsthead

%%  The running header definition.                                  %%
%	\hline
%	\multicolumn{3}{l}{\ldots\small\slshape continued} \\ \hline
%	\multicolumn{1}{c}{colTag}	%Column 1
%	&\multicolumn{1}{c}{colTag}	%Column 2
%	&\multicolumn{1}{c}{colTag}	\\ \hline %Last column
%	\endhead

%%  The running footer definition.                                  %%
%	\hline
%	\multicolumn{3}{r}{\small\slshape continued\ldots} \\
%	\endfoot

%%  The ending footer definition.                                   %%
%	\multicolumn{3}{c}{That's all folks} \\ \hline 
%	\endlastfoot
%%%%%%%%%%%%%%%%%%%%%%%%%%%%%%%%%%%%%%%%%%%%%%%%%%%%%%%%%%%%%%%%%%%%%%

\hhline{|-|-|-}
	 \multicolumn{1}{|p{\gnumericColA}|}%
	{\gnumericPB{\centering}\gnumbox{\textbf{Component}}}
	&\multicolumn{1}{p{\gnumericColB}|}%
	{\gnumericPB{\centering}\gnumbox{\textbf{Value}}}
	&\multicolumn{1}{p{\gnumericColC}|}%
	{\gnumericPB{\centering}\gnumbox{\textbf{Quantity}}}
\\
\hhline{|---|}
	 \multicolumn{1}{|p{\gnumericColA}|}%
	{\gnumericPB{\centering}\gnumbox{Vaman}}
	&\multicolumn{1}{p{\gnumericColB}|}%
	{\gnumericPB{\centering}\gnumbox{LC}}
	&\multicolumn{1}{p{\gnumericColC}|}%
	{\gnumericPB{\centering}\gnumbox{1}}
\\
\hhline{|---|}
	 \multicolumn{1}{|p{\gnumericColA}|}%
	{\gnumericPB{\centering}\gnumbox{USB-UART}}
	&\multicolumn{1}{p{\gnumericColB}|}%
	{}
	&\multicolumn{1}{p{\gnumericColC}|}%
	{\gnumericPB{\centering}\gnumbox{1}}
\\
\hhline{|---|}
	 \multicolumn{1}{|p{\gnumericColA}|}%
	{\gnumericPB{\centering}\gnumbox{Jumper Wires}}
	&\multicolumn{1}{p{\gnumericColB}|}%
	{\gnumericPB{\centering}\gnumbox{Female-to-Female}}
	&\multicolumn{1}{p{\gnumericColC}|}%
	{\gnumericPB{\centering}\gnumbox{10}}
\\
\hhline{|---|}
	 \multicolumn{1}{|p{\gnumericColA}|}%
	{\gnumericPB{\centering}\gnumbox{Jumper Wires}}
	&\multicolumn{1}{p{\gnumericColB}|}%
	{\gnumericPB{\centering}\gnumbox{Male-to-Female}}
	&\multicolumn{1}{p{\gnumericColC}|}%
	{\gnumericPB{\centering}\gnumbox{10}}
\\
\hhline{|---|}
	 \multicolumn{1}{|p{\gnumericColA}|}%
	{\gnumericPB{\centering}\gnumbox{USB Cable}}
	&\multicolumn{1}{p{\gnumericColB}|}%
	{\gnumericPB{\centering}\gnumbox{Type-B}}
	&\multicolumn{1}{p{\gnumericColC}|}%
	{\gnumericPB{\centering}\gnumbox{1}}
\\
\hhline{|---|}
	 \multicolumn{1}{|p{\gnumericColA}|}%
	{\gnumericPB{\centering}\gnumbox{Seven-Segment Display}}
	&\multicolumn{1}{p{\gnumericColB}|}%
	{\gnumericPB{\centering}\gnumbox{Common Anode}}
	&\multicolumn{1}{p{\gnumericColC}|}%
	{\gnumericPB{\centering}\gnumbox{1}}
\\
\hhline{|---|}
	 \multicolumn{1}{|p{\gnumericColA}|}%
	{\gnumericPB{\centering}\gnumbox{Breadboard}}
	&\multicolumn{1}{p{\gnumericColB}|}%
	{}
	&\multicolumn{1}{p{\gnumericColC}|}%
	{\gnumericPB{\centering}\gnumbox{1}}
\\
\hhline{|-|-|-|}
\end{tabular}

\ifthenelse{\isundefined{\languageshorthands}}{}{\languageshorthands{\languagename}}
\gnumericTableEnd

    \caption{Components Required for Controlling the Seven-Segment Display via SPI.}
    \label{tab:esp32-m4-fpga-sevenseg-components}
\end{table}

\subsubsection{Connections}
\begin{table}[!ht]
    \centering
    %%%%%%%%%%%%%%%%%%%%%%%%%%%%%%%%%%%%%%%%%%%%%%%%%%%%%%%%%%%%%%%%%%%%%%
%%                                                                  %%
%%  This is the header of a LaTeX2e file exported from Gnumeric.    %%
%%                                                                  %%
%%  This file can be compiled as it stands or included in another   %%
%%  LaTeX document. The table is based on the longtable package so  %%
%%  the longtable options (headers, footers...) can be set in the   %%
%%  preamble section below (see PRAMBLE).                           %%
%%                                                                  %%
%%  To include the file in another, the following two lines must be %%
%%  in the including file:                                          %%
%%        \def\inputGnumericTable{}                                 %%
%%  at the beginning of the file and:                               %%
%%        \input{name-of-this-file.tex}                             %%
%%  where the table is to be placed. Note also that the including   %%
%%  file must use the following packages for the table to be        %%
%%  rendered correctly:                                             %%
%%    \usepackage[latin1]{inputenc}                                 %%
%%    \usepackage{color}                                            %%
%%    \usepackage{array}                                            %%
%%    \usepackage{longtable}                                        %%
%%    \usepackage{calc}                                             %%
%%    \usepackage{multirow}                                         %%
%%    \usepackage{hhline}                                           %%
%%    \usepackage{ifthen}                                           %%
%%  optionally (for landscape tables embedded in another document): %%
%%    \usepackage{lscape}                                           %%
%%                                                                  %%
%%%%%%%%%%%%%%%%%%%%%%%%%%%%%%%%%%%%%%%%%%%%%%%%%%%%%%%%%%%%%%%%%%%%%%



%%  This section checks if we are begin input into another file or  %%
%%  the file will be compiled alone. First use a macro taken from   %%
%%  the TeXbook ex 7.7 (suggestion of Han-Wen Nienhuys).            %%
\def\ifundefined#1{\expandafter\ifx\csname#1\endcsname\relax}


%%  Check for the \def token for inputed files. If it is not        %%
%%  defined, the file will be processed as a standalone and the     %%
%%  preamble will be used.                                          %%
\ifundefined{inputGnumericTable}

%%  We must be able to close or not the document at the end.        %%
	\def\gnumericTableEnd{\end{document}}


%%%%%%%%%%%%%%%%%%%%%%%%%%%%%%%%%%%%%%%%%%%%%%%%%%%%%%%%%%%%%%%%%%%%%%
%%                                                                  %%
%%  This is the PREAMBLE. Change these values to get the right      %%
%%  paper size and other niceties.                                  %%
%%                                                                  %%
%%%%%%%%%%%%%%%%%%%%%%%%%%%%%%%%%%%%%%%%%%%%%%%%%%%%%%%%%%%%%%%%%%%%%%

	\documentclass[12pt%
			  %,landscape%
                    ]{report}
       \usepackage[latin1]{inputenc}
       \usepackage{fullpage}
       \usepackage{color}
       \usepackage{array}
       \usepackage{longtable}
       \usepackage{calc}
       \usepackage{multirow}
       \usepackage{hhline}
       \usepackage{ifthen}

	\begin{document}


%%  End of the preamble for the standalone. The next section is for %%
%%  documents which are included into other LaTeX2e files.          %%
\else

%%  We are not a stand alone document. For a regular table, we will %%
%%  have no preamble and only define the closing to mean nothing.   %%
    \def\gnumericTableEnd{}

%%  If we want landscape mode in an embedded document, comment out  %%
%%  the line above and uncomment the two below. The table will      %%
%%  begin on a new page and run in landscape mode.                  %%
%       \def\gnumericTableEnd{\end{landscape}}
%       \begin{landscape}


%%  End of the else clause for this file being \input.              %%
\fi

%%%%%%%%%%%%%%%%%%%%%%%%%%%%%%%%%%%%%%%%%%%%%%%%%%%%%%%%%%%%%%%%%%%%%%
%%                                                                  %%
%%  The rest is the gnumeric table, except for the closing          %%
%%  statement. Changes below will alter the table's appearance.     %%
%%                                                                  %%
%%%%%%%%%%%%%%%%%%%%%%%%%%%%%%%%%%%%%%%%%%%%%%%%%%%%%%%%%%%%%%%%%%%%%%

\providecommand{\gnumericmathit}[1]{#1} 
%%  Uncomment the next line if you would like your numbers to be in %%
%%  italics if they are italizised in the gnumeric table.           %%
%\renewcommand{\gnumericmathit}[1]{\mathit{#1}}
\providecommand{\gnumericPB}[1]%
{\let\gnumericTemp=\\#1\let\\=\gnumericTemp\hspace{0pt}}
 \ifundefined{gnumericTableWidthDefined}
        \newlength{\gnumericTableWidth}
        \newlength{\gnumericTableWidthComplete}
        \newlength{\gnumericMultiRowLength}
        \global\def\gnumericTableWidthDefined{}
 \fi
%% The following setting protects this code from babel shorthands.  %%
 \ifthenelse{\isundefined{\languageshorthands}}{}{\languageshorthands{english}}
%%  The default table format retains the relative column widths of  %%
%%  gnumeric. They can easily be changed to c, r or l. In that case %%
%%  you may want to comment out the next line and uncomment the one %%
%%  thereafter                                                      %%
\providecommand\gnumbox{\makebox[0pt]}
%%\providecommand\gnumbox[1][]{\makebox}

%% to adjust positions in multirow situations                       %%
\setlength{\bigstrutjot}{\jot}
\setlength{\extrarowheight}{\doublerulesep}

%%  The \setlongtables command keeps column widths the same across  %%
%%  pages. Simply comment out next line for varying column widths.  %%
\setlongtables

\setlength\gnumericTableWidth{%
	68pt+%
	76pt+%
	78pt+%
	70pt+%
0pt}
\def\gumericNumCols{4}
\setlength\gnumericTableWidthComplete{\gnumericTableWidth+%
         \tabcolsep*\gumericNumCols*2+\arrayrulewidth*\gumericNumCols}
\ifthenelse{\lengthtest{\gnumericTableWidthComplete > \linewidth}}%
         {\def\gnumericScale{1*\ratio{\linewidth-%
                        \tabcolsep*\gumericNumCols*2-%
                        \arrayrulewidth*\gumericNumCols}%
{\gnumericTableWidth}}}%
{\def\gnumericScale{1}}

%%%%%%%%%%%%%%%%%%%%%%%%%%%%%%%%%%%%%%%%%%%%%%%%%%%%%%%%%%%%%%%%%%%%%%
%%                                                                  %%
%% The following are the widths of the various columns. We are      %%
%% defining them here because then they are easier to change.       %%
%% Depending on the cell formats we may use them more than once.    %%
%%                                                                  %%
%%%%%%%%%%%%%%%%%%%%%%%%%%%%%%%%%%%%%%%%%%%%%%%%%%%%%%%%%%%%%%%%%%%%%%

\ifthenelse{\isundefined{\gnumericColA}}{\newlength{\gnumericColA}}{}\settowidth{\gnumericColA}{\begin{tabular}{@{}p{68pt*\gnumericScale}@{}}x\end{tabular}}
\ifthenelse{\isundefined{\gnumericColB}}{\newlength{\gnumericColB}}{}\settowidth{\gnumericColB}{\begin{tabular}{@{}p{76pt*\gnumericScale}@{}}x\end{tabular}}
\ifthenelse{\isundefined{\gnumericColC}}{\newlength{\gnumericColC}}{}\settowidth{\gnumericColC}{\begin{tabular}{@{}p{78pt*\gnumericScale}@{}}x\end{tabular}}
\ifthenelse{\isundefined{\gnumericColD}}{\newlength{\gnumericColD}}{}\settowidth{\gnumericColD}{\begin{tabular}{@{}p{70pt*\gnumericScale}@{}}x\end{tabular}}

\begin{tabular}[c]{%
	b{\gnumericColA}%
	b{\gnumericColB}%
	b{\gnumericColC}%
	b{\gnumericColD}%
	}

%%%%%%%%%%%%%%%%%%%%%%%%%%%%%%%%%%%%%%%%%%%%%%%%%%%%%%%%%%%%%%%%%%%%%%
%%  The longtable options. (Caption, headers... see Goosens, p.124) %%
%	\caption{The Table Caption.}             \\	%
% \hline	% Across the top of the table.
%%  The rest of these options are table rows which are placed on    %%
%%  the first, last or every page. Use \multicolumn if you want.    %%

%%  Header for the first page.                                      %%
%	\multicolumn{4}{c}{The First Header} \\ \hline 
%	\multicolumn{1}{c}{colTag}	%Column 1
%	&\multicolumn{1}{c}{colTag}	%Column 2
%	&\multicolumn{1}{c}{colTag}	%Column 3
%	&\multicolumn{1}{c}{colTag}	\\ \hline %Last column
%	\endfirsthead

%%  The running header definition.                                  %%
%	\hline
%	\multicolumn{4}{l}{\ldots\small\slshape continued} \\ \hline
%	\multicolumn{1}{c}{colTag}	%Column 1
%	&\multicolumn{1}{c}{colTag}	%Column 2
%	&\multicolumn{1}{c}{colTag}	%Column 3
%	&\multicolumn{1}{c}{colTag}	\\ \hline %Last column
%	\endhead

%%  The running footer definition.                                  %%
%	\hline
%	\multicolumn{4}{r}{\small\slshape continued\ldots} \\
%	\endfoot

%%  The ending footer definition.                                   %%
%	\multicolumn{4}{c}{That's all folks} \\ \hline 
%	\endlastfoot
%%%%%%%%%%%%%%%%%%%%%%%%%%%%%%%%%%%%%%%%%%%%%%%%%%%%%%%%%%%%%%%%%%%%%%

\hhline{|-|-|-~}
	 \multicolumn{1}{|p{\gnumericColA}|}%
	{\gnumericPB{\centering}\gnumbox{\textbf{Pin}}}
	&\multicolumn{1}{p{\gnumericColB}|}%
	{\gnumericPB{\centering}\gnumbox{\textbf{Vaman-ESP32}}}
	&\multicolumn{1}{p{\gnumericColC}|}%
	{\gnumericPB{\centering}\gnumbox{\textbf{Vaman-Pygmy}}}
	&
\\
\hhline{|---|~}
	 \multicolumn{1}{|p{\gnumericColA}|}%
	{\gnumericPB{\centering}\gnumbox{CS/SS}}
	&\multicolumn{1}{p{\gnumericColB}|}%
	{\gnumericPB{\centering}\gnumbox{27}}
	&\multicolumn{1}{p{\gnumericColC}|}%
	{\gnumericPB{\centering}\gnumbox{20}}
	&
\\
\hhline{|---|~}
	 \multicolumn{1}{|p{\gnumericColA}|}%
	{\gnumericPB{\centering}\gnumbox{CIPO/MISO}}
	&\multicolumn{1}{p{\gnumericColB}|}%
	{\gnumericPB{\centering}\gnumbox{19}}
	&\multicolumn{1}{p{\gnumericColC}|}%
	{\gnumericPB{\centering}\gnumbox{17}}
	&
\\
\hhline{|---|~}
	 \multicolumn{1}{|p{\gnumericColA}|}%
	{\gnumericPB{\centering}\gnumbox{COPI/MOSi}}
	&\multicolumn{1}{p{\gnumericColB}|}%
	{\gnumericPB{\centering}\gnumbox{18}}
	&\multicolumn{1}{p{\gnumericColC}|}%
	{\gnumericPB{\centering}\gnumbox{19}}
	&
\\
\hhline{|---|~}
	 \multicolumn{1}{|p{\gnumericColA}|}%
	{\gnumericPB{\centering}\gnumbox{SCLK}}
	&\multicolumn{1}{p{\gnumericColB}|}%
	{\gnumericPB{\centering}\gnumbox{5}}
	&\multicolumn{1}{p{\gnumericColC}|}%
	{\gnumericPB{\centering}\gnumbox{16}}
	&
\\
\hhline{|-|-|-|~}
\end{tabular}

\ifthenelse{\isundefined{\languageshorthands}}{}{\languageshorthands{\languagename}}
\gnumericTableEnd

    \caption{Connections to Establish SPI Between Vaman-ESP32 and Vaman-Pygmy.}
    \label{tab:esp32-m4-fpga-sevenseg-vaman-connections}
\end{table}

\begin{table}[!ht]
    \centering
    %%%%%%%%%%%%%%%%%%%%%%%%%%%%%%%%%%%%%%%%%%%%%%%%%%%%%%%%%%%%%%%%%%%%%%
%%                                                                  %%
%%  This is the header of a LaTeX2e file exported from Gnumeric.    %%
%%                                                                  %%
%%  This file can be compiled as it stands or included in another   %%
%%  LaTeX document. The table is based on the longtable package so  %%
%%  the longtable options (headers, footers...) can be set in the   %%
%%  preamble section below (see PRAMBLE).                           %%
%%                                                                  %%
%%  To include the file in another, the following two lines must be %%
%%  in the including file:                                          %%
%%        \def\inputGnumericTable{}                                 %%
%%  at the beginning of the file and:                               %%
%%        \input{name-of-this-file.tex}                             %%
%%  where the table is to be placed. Note also that the including   %%
%%  file must use the following packages for the table to be        %%
%%  rendered correctly:                                             %%
%%    \usepackage[latin1]{inputenc}                                 %%
%%    \usepackage{color}                                            %%
%%    \usepackage{array}                                            %%
%%    \usepackage{longtable}                                        %%
%%    \usepackage{calc}                                             %%
%%    \usepackage{multirow}                                         %%
%%    \usepackage{hhline}                                           %%
%%    \usepackage{ifthen}                                           %%
%%  optionally (for landscape tables embedded in another document): %%
%%    \usepackage{lscape}                                           %%
%%                                                                  %%
%%%%%%%%%%%%%%%%%%%%%%%%%%%%%%%%%%%%%%%%%%%%%%%%%%%%%%%%%%%%%%%%%%%%%%



%%  This section checks if we are begin input into another file or  %%
%%  the file will be compiled alone. First use a macro taken from   %%
%%  the TeXbook ex 7.7 (suggestion of Han-Wen Nienhuys).            %%
\def\ifundefined#1{\expandafter\ifx\csname#1\endcsname\relax}


%%  Check for the \def token for inputed files. If it is not        %%
%%  defined, the file will be processed as a standalone and the     %%
%%  preamble will be used.                                          %%
\ifundefined{inputGnumericTable}

%%  We must be able to close or not the document at the end.        %%
	\def\gnumericTableEnd{\end{document}}


%%%%%%%%%%%%%%%%%%%%%%%%%%%%%%%%%%%%%%%%%%%%%%%%%%%%%%%%%%%%%%%%%%%%%%
%%                                                                  %%
%%  This is the PREAMBLE. Change these values to get the right      %%
%%  paper size and other niceties.                                  %%
%%                                                                  %%
%%%%%%%%%%%%%%%%%%%%%%%%%%%%%%%%%%%%%%%%%%%%%%%%%%%%%%%%%%%%%%%%%%%%%%

	\documentclass[12pt%
			  %,landscape%
                    ]{report}
       \usepackage[latin1]{inputenc}
       \usepackage{fullpage}
       \usepackage{color}
       \usepackage{array}
       \usepackage{longtable}
       \usepackage{calc}
       \usepackage{multirow}
       \usepackage{hhline}
       \usepackage{ifthen}

	\begin{document}


%%  End of the preamble for the standalone. The next section is for %%
%%  documents which are included into other LaTeX2e files.          %%
\else

%%  We are not a stand alone document. For a regular table, we will %%
%%  have no preamble and only define the closing to mean nothing.   %%
    \def\gnumericTableEnd{}

%%  If we want landscape mode in an embedded document, comment out  %%
%%  the line above and uncomment the two below. The table will      %%
%%  begin on a new page and run in landscape mode.                  %%
%       \def\gnumericTableEnd{\end{landscape}}
%       \begin{landscape}


%%  End of the else clause for this file being \input.              %%
\fi

%%%%%%%%%%%%%%%%%%%%%%%%%%%%%%%%%%%%%%%%%%%%%%%%%%%%%%%%%%%%%%%%%%%%%%
%%                                                                  %%
%%  The rest is the gnumeric table, except for the closing          %%
%%  statement. Changes below will alter the table's appearance.     %%
%%                                                                  %%
%%%%%%%%%%%%%%%%%%%%%%%%%%%%%%%%%%%%%%%%%%%%%%%%%%%%%%%%%%%%%%%%%%%%%%

\providecommand{\gnumericmathit}[1]{#1} 
%%  Uncomment the next line if you would like your numbers to be in %%
%%  italics if they are italizised in the gnumeric table.           %%
%\renewcommand{\gnumericmathit}[1]{\mathit{#1}}
\providecommand{\gnumericPB}[1]%
{\let\gnumericTemp=\\#1\let\\=\gnumericTemp\hspace{0pt}}
 \ifundefined{gnumericTableWidthDefined}
        \newlength{\gnumericTableWidth}
        \newlength{\gnumericTableWidthComplete}
        \newlength{\gnumericMultiRowLength}
        \global\def\gnumericTableWidthDefined{}
 \fi
%% The following setting protects this code from babel shorthands.  %%
 \ifthenelse{\isundefined{\languageshorthands}}{}{\languageshorthands{english}}
%%  The default table format retains the relative column widths of  %%
%%  gnumeric. They can easily be changed to c, r or l. In that case %%
%%  you may want to comment out the next line and uncomment the one %%
%%  thereafter                                                      %%
\providecommand\gnumbox{\makebox[0pt]}
%%\providecommand\gnumbox[1][]{\makebox}

%% to adjust positions in multirow situations                       %%
\setlength{\bigstrutjot}{\jot}
\setlength{\extrarowheight}{\doublerulesep}

%%  The \setlongtables command keeps column widths the same across  %%
%%  pages. Simply comment out next line for varying column widths.  %%
\setlongtables

\setlength\gnumericTableWidth{%
	120pt+%
	75pt+%
	70pt+%
0pt}
\def\gumericNumCols{3}
\setlength\gnumericTableWidthComplete{\gnumericTableWidth+%
         \tabcolsep*\gumericNumCols*2+\arrayrulewidth*\gumericNumCols}
\ifthenelse{\lengthtest{\gnumericTableWidthComplete > \linewidth}}%
         {\def\gnumericScale{1*\ratio{\linewidth-%
                        \tabcolsep*\gumericNumCols*2-%
                        \arrayrulewidth*\gumericNumCols}%
{\gnumericTableWidth}}}%
{\def\gnumericScale{1}}

%%%%%%%%%%%%%%%%%%%%%%%%%%%%%%%%%%%%%%%%%%%%%%%%%%%%%%%%%%%%%%%%%%%%%%
%%                                                                  %%
%% The following are the widths of the various columns. We are      %%
%% defining them here because then they are easier to change.       %%
%% Depending on the cell formats we may use them more than once.    %%
%%                                                                  %%
%%%%%%%%%%%%%%%%%%%%%%%%%%%%%%%%%%%%%%%%%%%%%%%%%%%%%%%%%%%%%%%%%%%%%%

\ifthenelse{\isundefined{\gnumericColA}}{\newlength{\gnumericColA}}{}\settowidth{\gnumericColA}{\begin{tabular}{@{}p{120pt*\gnumericScale}@{}}x\end{tabular}}
\ifthenelse{\isundefined{\gnumericColB}}{\newlength{\gnumericColB}}{}\settowidth{\gnumericColB}{\begin{tabular}{@{}p{75pt*\gnumericScale}@{}}x\end{tabular}}
\ifthenelse{\isundefined{\gnumericColC}}{\newlength{\gnumericColC}}{}\settowidth{\gnumericColC}{\begin{tabular}{@{}p{70pt*\gnumericScale}@{}}x\end{tabular}}

\begin{tabular}[c]{%
	b{\gnumericColA}%
	b{\gnumericColB}%
	b{\gnumericColC}%
	}

%%%%%%%%%%%%%%%%%%%%%%%%%%%%%%%%%%%%%%%%%%%%%%%%%%%%%%%%%%%%%%%%%%%%%%
%%  The longtable options. (Caption, headers... see Goosens, p.124) %%
%	\caption{The Table Caption.}             \\	%
% \hline	% Across the top of the table.
%%  The rest of these options are table rows which are placed on    %%
%%  the first, last or every page. Use \multicolumn if you want.    %%

%%  Header for the first page.                                      %%
%	\multicolumn{3}{c}{The First Header} \\ \hline 
%	\multicolumn{1}{c}{colTag}	%Column 1
%	&\multicolumn{1}{c}{colTag}	%Column 2
%	&\multicolumn{1}{c}{colTag}	\\ \hline %Last column
%	\endfirsthead

%%  The running header definition.                                  %%
%	\hline
%	\multicolumn{3}{l}{\ldots\small\slshape continued} \\ \hline
%	\multicolumn{1}{c}{colTag}	%Column 1
%	&\multicolumn{1}{c}{colTag}	%Column 2
%	&\multicolumn{1}{c}{colTag}	\\ \hline %Last column
%	\endhead

%%  The running footer definition.                                  %%
%	\hline
%	\multicolumn{3}{r}{\small\slshape continued\ldots} \\
%	\endfoot

%%  The ending footer definition.                                   %%
%	\multicolumn{3}{c}{That's all folks} \\ \hline 
%	\endlastfoot
%%%%%%%%%%%%%%%%%%%%%%%%%%%%%%%%%%%%%%%%%%%%%%%%%%%%%%%%%%%%%%%%%%%%%%

\hhline{|-|-~}
	 \multicolumn{1}{|p{\gnumericColA}|}%
	{\gnumericPB{\centering}\gnumbox{\textbf{Seven-Segment Display}}}
	&\multicolumn{1}{p{\gnumericColB}|}%
	{\gnumericPB{\centering}\gnumbox{\textbf{Vaman-Pygmy}}}
	&
\\
\hhline{|--|~}
	 \multicolumn{1}{|p{\gnumericColA}|}%
	{\gnumericPB{\centering}\gnumbox{a}}
	&\multicolumn{1}{p{\gnumericColB}|}%
	{\gnumericPB{\centering}\gnumbox{1}}
	&
\\
\hhline{|--|~}
	 \multicolumn{1}{|p{\gnumericColA}|}%
	{\gnumericPB{\centering}\gnumbox{b}}
	&\multicolumn{1}{p{\gnumericColB}|}%
	{\gnumericPB{\centering}\gnumbox{2}}
	&
\\
\hhline{|--|~}
	 \multicolumn{1}{|p{\gnumericColA}|}%
	{\gnumericPB{\centering}\gnumbox{c}}
	&\multicolumn{1}{p{\gnumericColB}|}%
	{\gnumericPB{\centering}\gnumbox{3}}
	&
\\
\hhline{|--|~}
	 \multicolumn{1}{|p{\gnumericColA}|}%
	{\gnumericPB{\centering}\gnumbox{d}}
	&\multicolumn{1}{p{\gnumericColB}|}%
	{\gnumericPB{\centering}\gnumbox{4}}
	&
\\
\hhline{|--|~}
	 \multicolumn{1}{|p{\gnumericColA}|}%
	{\gnumericPB{\centering}\gnumbox{e}}
	&\multicolumn{1}{p{\gnumericColB}|}%
	{\gnumericPB{\centering}\gnumbox{5}}
	&
\\
\hhline{|--|~}
	 \multicolumn{1}{|p{\gnumericColA}|}%
	{\gnumericPB{\centering}\gnumbox{f}}
	&\multicolumn{1}{p{\gnumericColB}|}%
	{\gnumericPB{\centering}\gnumbox{6}}
	&
\\
\hhline{|--|~}
	 \multicolumn{1}{|p{\gnumericColA}|}%
	{\gnumericPB{\centering}\gnumbox{g}}
	&\multicolumn{1}{p{\gnumericColB}|}%
	{\gnumericPB{\centering}\gnumbox{7}}
	&
\\
\hhline{|--|~}
	 \multicolumn{1}{|p{\gnumericColA}|}%
	{\gnumericPB{\centering}\gnumbox{COM}}
	&\multicolumn{1}{p{\gnumericColB}|}%
	{\gnumericPB{\centering}\gnumbox{3v3}}
	&
\\
\hhline{|-|-|~}
\end{tabular}

\ifthenelse{\isundefined{\languageshorthands}}{}{\languageshorthands{\languagename}}
\gnumericTableEnd

    \caption{Connections to Interface Seven-Segment Display with Vaman-Pygmy.}
    \label{tab:esp32-m4-fpga-sevenseg-display-connections}
\end{table}

\subsubsection{Building}
\begin{enumerate}
    \item Build the PlatformIO project at
    \begin{lstlisting}
inter-chip/esp32-m4-fpga/sevenseg/codes/esp32
    \end{lstlisting}
    \item Flash the project .bin file using USB-UART connected to the 
    Vaman-ESP32, via PlatformIO or ArduinoDroid.
    \item Build the FPGA project .bin file by entering the following commands at
    a terminal window.
    \begin{lstlisting}
# The following variable can be sourced from .shellrc or .venv/bin/activate
export QORC_SDK_PATH=/path/to/pygmy-sdk
cd inter-chip/esp32-fpga/sevenseg/codes/fpga
make
    \end{lstlisting}
    \item On building successfully, the .bin file is generated at
    \begin{lstlisting}
inter-chip/esp32-fpga/sevenseg/codes/fpga/rtl/AL4S3B_FPGA_Top.bin
    \end{lstlisting}
    \item Edit the variable PROJ\_ROOT to point to the location of the pygmy-sdk
    folder on your system in the file
    \begin{lstlisting}
inter-chip/esp32-m4-fpga/sevenseg/codes/m4/GCC_Project/config.mk
    \end{lstlisting}
    \item Build the M4 project .bin file by entering the following commands at a
    terminal window.
    \begin{lstlisting}
cd inter-chip/esp32-m4-fpga/sevenseg/codes/m4/GCC_Project
make -j4
    \end{lstlisting}
    \item Flash both FPGA and M4 .bin files to the Vaman-Pygmy by resetting it
    to bootloader mode and entering the following command at a terminal window,
    \begin{lstlisting}
python3 /path/to/tinyfpga-programmer-gui.py --port /dev/ttyACMxx --mode m4-fpga --m4app /path/to/m4.bin --appfpga /path/to/AL4S3B_FPGA_Top.bin --reset
    \end{lstlisting}
    where /dev/ttyACMxx is the port at which the Vaman board is available. This
    can be obtained by inspecting the output of the following command (requires
    root/sudo privileges).
    \begin{lstlisting}
dmesg -w
    \end{lstlisting}
\end{enumerate}

\subsubsection{Demonstration}
\begin{enumerate}
    \item Find the IP address of the Vaman-ESP32 by inspecting the output of the
    serial terminal, or by typing at a terminal window
    \begin{lstlisting}
ifconfig
nmap -sn xx.yy.zz.0/24
    \end{lstlisting}
    where xx.yy.zz represents the first three octets of the IP address of your
    device on the WiFi network interface, found using \texttt{ifconfig}.
    \item Then, go to the following website and interact with the HTML form to
    display the required digit on the seven-segment display.
    \begin{lstlisting}
http://<VAMAN-IP>/sevenseg
    \end{lstlisting}
\end{enumerate}

\subsubsection{Exercises}
\begin{enumerate}
    \item Modify the ESP32 code to have one radion button for each segment.
    Optionally, add another radio button for the dot.
    \item Minimize the number of SPI transactions and the amount of data
    transmitted in each of them.
    \item (Optional) Use CSS to style the bland HTML form.
    \item (Optional) Use JavaScript to make the form reactive i.e., changes
    should be seen on toggling buttons or switches, and not when the user clicks
    to submit the form.
\end{enumerate}