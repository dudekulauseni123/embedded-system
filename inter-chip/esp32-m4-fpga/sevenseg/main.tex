\subsection{Seven-Segment Display}
This subsection illustrates the use of SPI Communication between the Vaman-ESP32
and the ARM Cortex-M4 subsystem onboard the Vaman-Pygmy by controlling a
seven-segment display via the ARM interface, where the GPIO pins are exposed by
the FPGA platform.

\subsubsection{Components}
The components required are listed in
\autoref{tab:esp32-fpga-sevenseg-components}.

\subsubsection{Connections}
The connections required on the Vaman board are listed in
\autoref{tab:esp32-fpga-led-connections}. The connections to interface the
seven-segment display are listed in
\autoref{tab:esp32-fpga-sevenseg-display-connections}.

\subsubsection{Building}
\begin{enumerate}
    \item Build the PlatformIO project at
    \begin{lstlisting}
inter-chip/esp32-m4-fpga/sevenseg/codes/esp32
    \end{lstlisting}
    \item Flash the project .bin file using USB-UART connected to the 
    Vaman-ESP32, via PlatformIO or ArduinoDroid.
    \item Build the FPGA project .bin file by entering the following commands at
    a terminal window.
    \begin{lstlisting}
# The following variable can be sourced from .shellrc or .venv/bin/activate
export QORC_SDK_PATH=/path/to/pygmy-sdk
cd inter-chip/esp32-fpga/sevenseg/codes/fpga
make
    \end{lstlisting}
    \item On building successfully, the .bin file is generated at
    \begin{lstlisting}
inter-chip/esp32-fpga/sevenseg/codes/fpga/rtl/AL4S3B_FPGA_Top.bin
    \end{lstlisting}
    \item Edit the variable PROJ\_ROOT to point to the location of the pygmy-sdk
    folder on your system in the file
    \begin{lstlisting}
inter-chip/esp32-m4-fpga/sevenseg/codes/m4/GCC_Project/config.mk
    \end{lstlisting}
    \item Build the M4 project .bin file by entering the following commands at a
    terminal window.
    \begin{lstlisting}
cd inter-chip/esp32-m4-fpga/sevenseg/codes/m4/GCC_Project
make -j4
    \end{lstlisting}
    \item Flash both FPGA and M4 .bin files to the Vaman-Pygmy by resetting it
    to bootloader mode and entering the following command at a terminal window,
    \begin{lstlisting}
python3 /path/to/tinyfpga-programmer-gui.py --port /dev/ttyACMxx --mode m4-fpga --m4app /path/to/m4.bin --appfpga /path/to/AL4S3B_FPGA_Top.bin --reset
    \end{lstlisting}
    where /dev/ttyACMxx is the port at which the Vaman board is available. This
    can be obtained by inspecting the output of the following command (requires
    root/sudo privileges).
    \begin{lstlisting}
dmesg -w
    \end{lstlisting}
\end{enumerate}

\subsubsection{Demonstration}
\begin{enumerate}[resume]
    \item Find the IP address of the Vaman-ESP32 by inspecting the output of the
    serial terminal, or by typing at a terminal window
    \begin{lstlisting}
ifconfig
nmap -sn xx.yy.zz.0/24
    \end{lstlisting}
    where xx.yy.zz represents the first three octets of the IP address of your
    device on the WiFi network interface, found using \texttt{ifconfig}.
    \item Then, go to the following website and interact with the HTML form to
    display the required digit on the seven-segment display.
    \begin{lstlisting}
http://<VAMAN-IP>/sevenseg
    \end{lstlisting}
\end{enumerate}

\subsubsection{Exercises}
\begin{enumerate}[resume]
    \item Modify the ESP32 code to have one radion button for each segment.
    Optionally, add another radio button for the dot.
    \item Minimize the number of SPI transactions and the amount of data
    transmitted in each of them.
    \item (Optional) Use CSS to style the bland HTML form.
    \item (Optional) Use JavaScript to make the form reactive i.e., changes
    should be seen on toggling buttons or switches, and not when the user clicks
    to submit the form.
\end{enumerate}