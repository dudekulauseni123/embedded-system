\begin{abstract}
Through this manual, we will learn how to setting up the vaman-ESP as a Master and  arduinos as a Slave using SPI protocol. The  unknown resistance are measured by using Arduino and sending those  resistance value to Vaman through SPI and displaying the unkwnown Resistance on ESP-Webserver.
\end{abstract}
\subsection{Components}
\numberwithin{equation}{enumi}
\numberwithin{figure}{enumi}
%\numberwithin{table}{enumi}
\numberwithin{table}{section}
\begin{table}[!h]
\centering
\input{./vaman/vaman-esp/SPI-Resistance/figs/components.tex}
\caption{Components}
\label{table:SPI-components}
\end{table}

\subsection{Connections}
\begin{enumerate}[label=\thesection.\arabic*.,ref=\thesection.\theenumi]
\numberwithin{equation}{enumi}
\numberwithin{figure}{enumi}
\numberwithin{table}{enumi}

	

\item
Connect the Vaman and Arduino as shown Table. \ref{Tab:SPI-Connection}.

%
\begin{table}[!ht]
\centering
\input{./vaman/vaman-esp/SPI-Resistance/figs/connection.tex}
\caption{Connections}
\label{Tab:SPI-Connection}
\end{table}
\item
The Vaman pin diagram is available in Fig. \ref{fig:vaman-pin_sheet}

\item Upload the following code to Arduino UNO

\begin{lstlisting}
vaman/vaman-esp/SPI-Resistance/codes/arduino
\end{lstlisting}
\end{enumerate}
\subsection{Measuring the resistance}
\begin{enumerate}[label=\thesection.\arabic*.,ref=\thesection.\theenumi]
\numberwithin{equation}{enumi}
\numberwithin{figure}{enumi}
\numberwithin{table}{enumi}

\item
Connect the 5V pin of the Vaman-ESP to an extreme pin of the Breadboard shown in Fig. \ref{fig:breadboard}.  Let this pin be $V_{cc}$.
\item
Connect the GND pin of the Vaman-ESP to the opposite extreme pin of the Breadboard.

%
%
\item
Let $R_1$ be the known resistor and $R_2$ be the unknown resistor.  Connect $R_1$ and $R_2$ in series such that $R_1$ is connected
to $V_{cc}$ and $R_2$ is connected to GND. Refer to Fig. \ref{fig:voltage_divider}
\item
Connect the junction between the two resistors to  the A0 pin on the esp32-GPIO36, which measures the  unknown resistance.
\item
Now Power the Vaman board
\item
Execute the following code after editing the wifi credentials
\begin{lstlisting}
vaman/vaman-esp/SPI-Resistance/codes/esp32
\end{lstlisting}
\end{enumerate}
\subsection{Displaying the Measured resistance on website}
\begin{enumerate}[label=\thesection.\arabic*.,ref=\thesection.\theenumi]
\numberwithin{equation}{enumi}
\numberwithin{figure}{enumi}
\numberwithin{table}{enumi}
\item The unknown resistance is measured and diplayed the measured resistance on the Vaman-ESP webserver.
\item After flashing the code to vaman-ESP, the board will be connected to the wifi credentials provided.
\item Now connect the same WiFi credentials to the mobile phone for accessing the IP address, which can be accessed by 
\begin{lstlisting}
ifconfig
nmap -sn 192.168.x.x/24
\end{lstlisting}
\item Change the IP address in the second command accordingly with the IP address provided by first command.
\item By the above commands the IP address of vaman-ESP will be diplayed.
\item Now the vaman-ESP will be hosting a webserver
\item Inorder to access the webserver type the IP address of the vaman-ESP in the web browser.
\item In the website loaded by the IP address of vaman-ESP the Unknown resitance is displayed as shown in Fig. \ref{fig:results1}
\begin{figure}[H]
\centering
\includegraphics[width=\columnwidth]{./vaman/vaman-esp/SPI-Resistance/figs/result.jpg}
\caption{Website}
\label{fig:results1}
\end{figure}
\end{enumerate}
