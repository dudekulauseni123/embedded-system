\section{I2C Communication Between Vaman-ESP32 and Arduino}
This section describes how to setup the Vaman-ESP32 as a Master and Arduino as a
slave using I2C protocol.
\subsection{Components}
\numberwithin{equation}{enumi}
\numberwithin{figure}{enumi}
\numberwithin{table}{section}

\begin{table}[!ht]
\centering
\input{vaman-esp32/i2c/figs/components7.tex}
\caption{Components}
\label{table:i2c-components}
\end{table}

\subsection{Setting up the Master and Slave}
\begin{enumerate}[label=\thesection.\arabic*.,ref=\thesection.\theenumi]
\numberwithin{equation}{enumi}
\numberwithin{figure}{enumi}
\numberwithin{table}{enumi}

\item
Connect the vaman-ESP pins to Arduino pins as per Table \ref{Table:1 Arduino-ESP}.
\begin{table}[!ht]
\centering
\input{vaman-esp32/i2c/figs/table1.tex}
\caption{}
\label{Table:1 Arduino-ESP}
\end{table}

\item Connect the vaman-ESP pins to LCD pins as per \ref{Table:1}.

\item The Vaman pin diagram is available in Fig. \ref{fig:vaman-pin_sheet}

\item
Configure Arduino Uno as a Slave using the following PlatformIO project.
\begin{lstlisting}
vaman-esp32/i2c/codes/I2C_Sender_Arduino
\end{lstlisting}
\item
Now configure vaman-ESP as a Master using the following PlatformIO project.
\begin{lstlisting}
vaman-esp32/i2c/codes/I2C_Reciever_ESP32
\end{lstlisting}

\end{enumerate}