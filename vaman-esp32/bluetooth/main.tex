This section describes how to control the seven-segment display through the
Dabble Android application using Bluetooth, and display an appropriate digit on 
the seven-segment display according to the controls in the Android app.

\subsection{Components}
\numberwithin{equation}{enumi}
\numberwithin{figure}{enumi}
\numberwithin{table}{section}

\begin{table}[!ht]
\centering
\input{vaman-esp32/bluetooth/tables/components.tex}
\caption{Components}
\label{table:ble-components}
\end{table}
\subsection{Connections}
\begin{enumerate}[label=\thesection.\arabic*.,ref=\thesection.\theenumi]
\numberwithin{equation}{enumi}
\numberwithin{figure}{enumi}
\numberwithin{table}{enumi}
\item Connect the Arduino-UART to VAMAN as per Table. \ref{tab:arduino-uart} and Figure \ref{fig:vaman/uart/1}.
\item Now, build the PlatformIO project at
\begin{lstlisting}
vaman-esp32/bluetooth/codes
\end{lstlisting}
\item Install the Dabble Android application and give the necessary permissions.
\item Connect the Bluetooth of Vaman-ESP32 to the mobile where the Bluetooth 
device name is labelled as ``MyEsp32''.
\item Open the Dabble application. Select gamepad option in the app and then select Digital Mode and connect it app to ESP-32 by connecting it ESP-32 bluetooth as shown in Figure \ref{fig:ble_app}.
\begin{figure}[!ht]
\centering
\includegraphics[width=\columnwidth]{vaman-esp32/bluetooth/figs/ble_app.jpg}
\caption{Dabble app Interface}
\label{fig:ble_app}
\end{figure}
\item Now connect the Seven Segment to the Vaman board according to the given Table. \ref{table:ble-connections}
\begin{table}[!ht]
\centering
\input{vaman-esp32/bluetooth/tables/connections.tex}
\caption{Connections}
\label{table:ble-connections}
\end{table}
\item Now you can observe the changes on sevensegment display for Start, Up, Down, Right and Left keys pressed on the Digital Mode on the android application
\end{enumerate}
